\documentclass[10pt]{article}

\usepackage{amsmath}
\usepackage{CJK}
\usepackage{indentfirst}

\begin{document}
\begin{CJK}{UTF8}{gbsn}

\title{lab1 实验报告}
\author{计83{ }朱文雷{ }2008011369}
\maketitle

\section{代码修改说明}
\begin{itemize}
\item 增加了文件 lock.h,其中定义了 Lock 结构体,包含 client 和 cond (类型为 pthread\_cond\_t) 两个成员。
\item 修改了 lock\_smain.cc 增加了 acquire 和 release 的函数注册。
\item lock\_server.h 增加了 acquire 和 release 的定义。增加了成员 server\_mutex,增加了 locks 成员,用来存储锁。
\item lock\_server.cc 构造函数增加了 server\_mutex 的初始化。增加了 acquire 和 release 函数。
\item lock\_client.cc 增加了 acquire 和 release 函数的实现。
\item 增加了 zdebug.h 用来调试输出。
\item 修改了 rpc/rpc.cc 文件的 checkduplicate\_and\_update 函数和 add\_reply 函数,实现了所有求的功能。
\end{itemize}

\section{程序思路说明}
\subsection{lock\_client}
lock\_client 的实现非常简单,直接调用 rpcc 的 call 函数即可,并传入正确的参数。

\subsection{lock\_server}
\subsubsection{acquire逻辑}
首先用一个 ScopedLock 把整个函数锁住。然后在 locks 中搜索 lid 这个 key,如果没有的话,说明这个锁不存在,那么直接为其分配一个锁即可。如果存在的话,就去检查这个锁对应的 client,如果 client 为正,说明锁被某一个 client 所占用,那么就一直去 wait 这个锁的 cond ,直到被 broadcast 唤醒,并发现 client 为 -1的时候,就可以获得这个锁了,把锁的 client 赋值为当前 client,然后返回。

\subsubsection{release 逻辑}
首先用一个 ScopedLock 把整个函数锁住。然后在 locks 中找这个锁,找到之后将其 client 置为 -1,并 broadcast 这个锁的 cond,然后返回即可。

\subsection{at most once 的实现}
\subsubsection{checkduplicate\_and\_update 函数}
首先根据 clt\_nonce 找到这个 client 对应的 window 链表。

然后进行 window forward 操作。在 window 不为空的情况下,对 window 链表从头开始遍历,如果当前元素的 xid 值小于等于 xid\_rep ,则删除它的 buf。当循环结束后,使用 erase 函数,把这些元素都删除。

如果 window 不为空,那么下面首先看 window 的第一个元素的 xid 是不是大于 这次请求的 xid ,如果是的话,则返回 FORGOTTEN。

如果 xid 的值小于 window 的末尾元素的 xid 的话,则对链表进行遍历,找到后检查其 cb\_present 值,如果为 true 则返回 INPROGRESS, 否则的话,将 b 和 sz 赋值后返回 DONE
。

如果上面这么多检查仍然没有 return 的话,说明肯定是一个新的请求。首先构造 reply\_t 元素。然后遍历链表,找到他应该待的位置后插入链表,最后返回 NEW 即可。

\subsubsection{add\_reply 逻辑}
首先找到当前 clt\_nonce 对应的 链表。
然后对链表进行遍历,找到对应元素后,将 buf 和 sz 赋上传进来的值,并将 cb\_present 标记为 false。然后返回即可。

\end{CJK}
\end{document}
